\section{The Language Identification Problem}
\label{sec:lid}

\subsection{Language Identification as the key to speech tasks}

Current generation language system are already available and in production for a number of industries. To make customer support more helpful and user friendly some call centers apply language identification systems to route customers to native speaking

Law enforcement agency deploy language identification systems for identifying suspect locations based on speech features and accents in telephony data. \todo{Source}. To foster research in this field DARPA established the TIMIT dataset for language identification of North American speakers. \todo{source}

At the time of this writing we had to acknowledge that not even Google Translate has automated language detection to determine the input language for a speech sample. Automated input language detection is available for written texts but Google Translate's voice input feature is only available for use after manually selection an input language. We believe that having an automated language detection system could be extremely helpful to users. 

Not only would this improve mobile and hands free interaction with the tool but also bring us closer to a



\subsection{Task Specification in This Thesis}
The system proposed in the thesis is language identification (\ac{lid}) system for classifying the languages of a given audio recording. We make use of human voices recordings to train the  system. We evaluate the suitability of convolutional neural networks for a LID system. We extend this approach with a recurrent neural network to form a hybrid network known as convolutional recurrent neural networks (\ac{crnn}). The system 's performance is evaluated on a set of news broadcasts and speeches made by members of the European Parliament. We further assess the system's robustness to noisy environments and background music. We state the following hypotheses:

\begin{enumerate}
	\item Convolutional neural networks can be successfully used for language identification tasks with high accuracy.
	\item Spectrogram images are a suitable input representation for learning audio features.
	\item Convolutional recurrent neural networks improve the classification accuracy for our LID task compared to a CNN based approach.
\end{enumerate}


