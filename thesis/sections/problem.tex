\section{The Language Identification Problem}
\label{sec:lid}

\emph{Automatic language identification} (\ac{lid}) is the process of determining the language spoken in an audio recording. The language identification systems proposed in this thesis consume audio recordings and use deep learning techniques to perform an automated classification of the recordings' source language.
Language identification is often the first step in a spoken-language processing pipelines. Automatic language identification systems are employed by a wide variety of applications from industry and research to entertainment.
In the following, multiple examples for language identification systems are presented.

\subsection{Language Identification as the Key to Speech Tasks}
In contrast to LID systems, \emph{automatic speech recognition} (\ac{asr}) is the task of transcribing spoken language into readable text, sometimes also known as \emph{speech-to-text systems}. Selecting the correct input language for these systems is crucial for transcribing single letters into meaningful words and adhering to the correct grammar. Many commercially available ASR system require manually setting the input language and would greatly benefit from an automated language identification system as part of their processing pipeline.

Many call center operators benefit from LID systems by automatically routing telephone calls in order to connect a caller with a suitable native speaker. This approach is used both for customer care hotlines (connecting callers with help desk agents) as well as government agencies, such as emergency telephone services. Muthusamy et al. report that manually matching emergency calls to US law enforcement agencies with native speakers involves a significant delay of up to three minutes. Automatic language identification systems speed up this process and efficiently support human agents~\cite{muthusamy1994reviewing}.

Similar to language identification, some research is focused on dialect detection. To foster research in this field, DARPA established the TIMIT dataset for dialect identification of North American speakers~\cite{garofolo1993darpa}. The dataset features eight major North American dialects and has long been the default corpus for comparative research on LID systems.
Recently, Germany's Federal Office for Migration and Refugees (BAMF) announced plans for using dialect identification systems as an additional resource in identifying a person's origin.\footnote{\url{http://www.theverge.com/2017/3/17/14956532/germany-refugee-voice-analysis-dialect-speech-software}, accessed 13 April 2017}

Language identification is also the first step in automated translation tasks. At the time of writing this thesis, we discovered that not even the Google Translate mobile app has automated language detection to determine the input language for speech input. Automated input language detection is available for written texts, but Google Translate's voice input feature is only available for use after manually selecting an input language. We believe that having an automated language detection system is extremely helpful to users.

An ever-increasing number of people connect to the internet or communicate with each other using their smartphones. While touchscreen input is complicated in some situations, the demand for voice interaction grows steadily. Many tasks can already be solved through voice input. This hands-free, voice-enabled interaction is a trend that we see in other industries such as automative computing as well.

Automatic LID systems and machine intelligence are also helpful for some entertainment companies. When we started working on this thesis, we were briefly inspired by the challenges of the Berlin-based start-up Dubsmash.\footnote{\url{https://www.dubsmash.com/}, accessed 13 April 2017} Their app lets users create a mash-up of a large variety of existing songs, movie quotes, or other voice snippets with a ten-second video recording of the user. The resulting video clip can be shared with friends and usually features a funny and personal reinterpretation of the audio source's original context. The success of the app is directly proportional to users' interest in the offered sound clips. In some cases, however, users are offered clips in foreign languages, which are often perceived as not funny or inappropriate. In this case, a LID system could be used to classify the language of the millions of audio snippets available in their library. Consequently, only sounds in the users' native languages could be recommended to the users.


\subsection{Task Specification in This Thesis}
\label{sec:task-specification}
In this thesis, we propose a language identification system for classifying the languages of given audio recordings. Our system is trained using human voice recordings. We evaluate the suitability of convolutional neural networks for a LID system. Finally, this approach is extended with a recurrent neural network to compose a hybrid network known as \emph{convolutional recurrent neural network} (\ac{crnn}). The system's performance is evaluated on a set of news broadcasts and speeches made by members of the European Parliament. We further assess the system's robustness to noisy environments and background music. This thesis states the following hypotheses:

\begin{enumerate}
	\item Convolutional neural networks are an effective, high-accuracy solution for language identification tasks.
	\item Spectrogram images are a suitable input representation for learning audio features.
	\item Convolutional recurrent neural networks improve the classification accuracy for our LID task compared to a plain, CNN-based approach.
\end{enumerate}


