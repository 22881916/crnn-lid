\section{Introduction}

\subsection{Language Identification as the key to speech tasks}


Current generation language system are already available and in production for a number of industries. To make customer support more helpful and user friendly some call centers apply language identification systems to route customers to native speaking

Law enforcement agency deploy language identification systems for identifying suspect locations based on speech features and accents in telephony data. \todo{Source}. To foster research in this field DARPA established the TIMIT dataset for language identification of North American speakers. \todo{source}

At the time of this writing we had to acknowledge that not even Google Translate has automated language detection to determine the input language for a speech sample. Automated input language detection is available for written texts but Google Translate's voice input feature is only available for use after manually selection an input language. We believe that having an automated language detection system could be extremely helpful to users. 

Not only would this improve mobile and hands free interaction with the tool but also bring us closer to a



\subsection{Contributions}
In this thesis we present an approach to language identification systems. We approach this task by using deep learning techniques. We transfer the given audio classification problem into an image based task to apply image recognition algorithms. Our contributions can be summarized as follows:
\begin{itemize}
	\item We investigate the suitability of convolutional neural networks to the task of language identification. We propose a hybrid network, combing the descriptive powers of convolutional neural networks with the ability of recurrent neural networks to learn time series. This approach is called convolutional recurrent neural network (CRNN). 
	\item We implemented a Python system of such a CRNN approach using the deep learning framework Keras and TensorFlow. 
	\item In train our system we gather our own large scale dataset of audio recordings. We explain how we obtained more than a thousand hours of human speech data suitable for our task.
	\item We assessed several machine learning metrics on our system with respect to our test data. Furthermore, we investigated the influence of noisy environments on our system. We discuss the system's ability to differentiate between several languages and how such a system can be extended to more languages.
	\item To showcase our system we developed a web service demo application using our best performing model. Further, we published said model for use by others.
\end{itemize} 


\subsection{Outline of the Thesis}
This thesis is structure as follows: In chapter \ref{sec:lid} we introduce the language identification problem and state our research hypotheses. Chapter \ref{sec:theoretical_background} explains the theoretical background of the deep learning techniques used in this thesis. In chapter \ref{sec:results_eu}} , we will 
