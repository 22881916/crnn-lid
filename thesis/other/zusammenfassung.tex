\section*{\LARGE Zusammenfassung}
Computern mit Spracheingabem\"oglichkeiten sind immer h\"aufiger anzutreffen. Deshalb sind Nutzer vermehrt auf verl\"assliche Spracherkennungsalgorithmen angewiesen. Dabei ist automatische Sprachidentifizierung der erste und wichtigste Schritt f\"ur die eigentliche Spracherkennung. Ohne automatische Sprachidentifizierung sind alle weiteren Erkennungsschritt nutzlos, da weder Sprachkl\"ange richtig verarbeitet noch Grammatikregeln ordentlich angewandt werden.

Ein zweiter aktueller Trend in der Informatik ist die erfolgreiche Anwendung von tiefen neuronalen Netzwerken f\"ur eine Vielzahl von Aufgabenstellungen. In dieser Arbeit wird ein hybrides, neuronales Netzwerkmodell zur automatischen Identifizierung unterschiedlicher Sprachen mithilfe von \emph{deep learning} Techniken vorgestellt. Im Speziellen werden mehrere \emph{convolutional recurrent neural network} Modelle auf menschliche Spracheingaben angewandt und deren Robustheit in verschiedenen Ger\"auschumgebungen evaluiert.

\emph{Convolutional neural networks} haben innerhalb der automatischen Bilderkennungsforschung beachtliche Durchbr\"uche erzielt. Deshalb wird unsere audiobasierte Forschungsfrage in die Bildverarbeitungsdom\"ane eingebettet. Dar\"uber hinaus basiert diese Arbeit auf deren etablierten Modellarchitekturen, wie beispielsweise das Inception Netzwerk\cite{szegedy2015going}. Dabei wird die Effektivit\"at von Spektrogrammbildern als zielf\"uhrendes Eingabemedium analysiert. Zus\"atzlich werden weitere Audiorepr\"asentationen und weiterf\"uhrende Publikationen vorgestellt.

Insbesondere die Verf\"ugbarkeit von gro{\ss}en Datens\"atzen kommen \emph{deep learning} Systemen zu Gute. Unsere Modelle werden auf mehr als \num{1000} Stunden an Sprachaufnahmen in sechs verschiedenen Sprachen trainiert: Englisch, Deutsch, Franz\"osisch, Spanisch, Mandarin und Russisch. Die Datengrundlage bilden Reden und Sitzungen des Europ\"aischen Parlaments, sowie YouTube Kan\"ale verschiedener Nachrichtensender, wie beispielsweise der BBC.

Unser leistungsf\"ahigstes \emph{convolutional recurrent neural network} Modell erreicht eine Erkennungsgenauigkeit und ein F-Ma{\ss} von \SI{96}{\percent} auf dem Nachrichtendatensatz. Unser Ansatz stellt eine konstante Verbesserung gegen\"uber allen getesteten \emph{convolutional neural networks} dar. Die Modelle werden zus\"atzlich in verschiedenen Szenarien mit k\"unstlich ver\"anderten Daten evaluiert: Dabei f\"ugen wir St\"orger\"ausche, wie Rauschen, Geknister und Hintergrundmusik zu den Daten hinzu und messen einen Genauigkeitsverlust von jeweils 5 Prozentpunkten (PP), 3 PP und 7 PP gegen\"uber den Originaldaten. Auf dem kleineren EU-Datensatz erreichen wir eine Genauigkeit und ein F-Ma{\ss} von \SI{98}{\percent}.